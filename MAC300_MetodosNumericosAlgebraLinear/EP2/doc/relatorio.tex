\input texbase

\titulo{Exercício Programa 2}
\materia{MAC300 - Métodos Numéricos da Álgebra Linear}

\aluno{Fernando Omar Aluani}{6797226}

\begin{document}
\cabecalho

\section{Filtros}

\subsection{Ajuste do Contraste - Equalização de Histograma}
\subsubsection{Implementação}
O filtro de contraste é implementado equalizando o histograma da imagem. Inicialmente o algoritmo percorre por cada pixel
da imagem para contar o nivel de cinza para construir o histograma. Após isso, a função de distribuição acumulada é 
calculada, armazenando seu resultado para cada nível de cinza em um vetor, e o valor mínimo em uma variável à parte.

Finalmente, a imagem processada é construída, iniciando seus pixeis com valor 0 e indo um-a-um calculando o novo valor
de cada pixel, normalizando o histograma.

\subsubsection{Vantagens e Desvantagens}
%Descreva vantagens e desvantagens de cada um dos métodos (em relação a desempenho computacional e desempenho como filtro)
\subsubsection{Exemplo}
\begin{figure}[htb]
    \centering
    \includegraphics[width=1.0\textwidth]{../contrast-final.jpg}
    \caption{O novo site do USPGameDev.}
    \label{fig:site_01}
\end{figure}

\subsection{Suavização por Média Ponderada - \textit{Blurring}}
\subsubsection{Implementação}
O filtro de \textit{blur} foi implementado usando a técnica de filtro de convolução, com uma máscara $3\times 3$ para tirar
a média ponderada dos pixeis ao redor do pixel sendo suavizado. O resultado da convolução da imagem já é o resultado
do processamento.

\subsubsection{Vantagens e Desvantagens}
%Descreva vantagens e desvantagens de cada um dos métodos (em relação a desempenho computacional e desempenho como filtro)
\subsubsection{Exemplo}
COLOCAR COMPARISON IMAGE HERE

\subsection{Aumento de Nitidez - \textit{Sharpening}}
\subsubsection{Implementação}
O filtro de \textit{sharpen} foi implementado usando a técnica de filtro de convolução, com uma máscara $3\times 3$
conhecida como \textit{8 neighbour Laplacian}. Usando a convolução na imagem original adquirimos o resultado do operador
laplaciano na imagem. O resultado final da imagem processada é a soma da imagem original com o laplaciano dela.

Caso o programa mostre a comparação das imagens (ver \textit{README}), ele irá mostrar também a imagem do operador 
laplaciano. 

\subsubsection{Vantagens e Desvantagens}
%Descreva vantagens e desvantagens de cada um dos métodos (em relação a desempenho computacional e desempenho como filtro)
\subsubsection{Exemplo}
COLOCAR COMPARISON IMAGE HERE

\end{document}