\documentclass[a4paper,11pt]{article}
\usepackage[T1]{fontenc}
\usepackage[utf8]{inputenc}
\usepackage{lmodern}

\title{EP1 MAC300 - Relatório}
\author{Fernando Omar Aluani (NUSP: 6797226)}

\begin{document}

\maketitle

\section{Modelando Som com Série de Fourier}

Podemos usar Séries de Fourier para modelar um som (uma onda sonora, no domínio do tempo) no domínio da frequência,
o que ajuda muito para analisar as propriedades do som (como harmônicos e notas).

O som originalmente está em tempo-contínuo, porém como não podemos analisá-lo dessa maneira, então primeiramente
amostramos o som para que tenhamos a representação do som em tempo-discreto. Então dividimos o som (em tempo-discreto)
em intervalos constantes, para analisarmos cada intervalo aplicando a Transformada Discreta de Fourier 
(DFT) em tempo-discreto.

A DFT devolve a representação desse intervalo do som no domínio da frequência - na forma de um
conjunto de ondas senoidais que somadas resultam na onda original do som, e a frequência de cada uma dessas senóides é
uma fração da frequência do som original. Assim, cada senóide corresponde a uma nota, e analisando as senóides podemos 
descobrir qual é a "mais forte" - que corresponde ao harmônico sendo tocado no som. A partir disso também podemos descobrir
qual é a frequência fundamental, e outras propriedades do som.

\section{Transcrição Musical}

A DFT retorna um vetor de números complexos $a + bi$. Cada posição desse vetor (e o complexo armazenado nela) corresponde
a uma senóide que compõe o som original. A posição no vetor indica a frequência da senóide (uma fração do \textit{framerate}
do som original), e o módulo do número complexo ($\sqrt{a^{2} + b^{2}}$) em tal posição indica a amplitude da senóide.

O EP, ao fazer a transcrição musical de um arquivo .wav, separa o conjunto de dados em intervalos (equivalentes a 0.1
segundos), e procura pelo maior pico (a maior amplitude) nesse intervalo, guardando essa amplitude (do pico) e a frequência
equivalente como sendo a "nota" do intervalo. Ao ir fazendo isso para cada intervalo do conjunto de dados, ele vai juntando
notas de frequência iguais (aumentando a duração da mesma, e salvando a variação da amplitude dela), ou criando notas novas
para frequências diferentes.

Após a análise, ao salvar as notas em um arquivo MIDI, o EP usa a frequência para determinar a nota MIDI mais próxima, e usa
os valores de amplitude que foram salvos para determinar a variação de volume na nota. Isso e o instrumento que escolhi para
salvar os MIDIs (Ocarina) produzem um arquivo MIDI que é praticamente igual ao .wav original (nos vários testes que fiz com 
os arquivos de exemplos disponibilizados pela monitora no PACA).

\section{Comparação dos Métodos de DFT}

O EP tem implementado 3 métodos de DFT, e pode executar qualquer combinação deles puramente para comparar a eficiência entre
eles. Quais métodos serão usados é definido por um argumento passado na linha de comando (vide o README).

Os métodos são:
\begin{itemize}
  \item \textbf{FFT}: método de \textit{fast-fourier transform} do NumPy.
  \item \textbf{Matrix}: método básico de multiplicação pela matriz de DFT. As colunas da matriz são guardadas como vetores do
  NumPy, e durante a aplicação da DFT (a multiplicação com o vetor de dados) uma função de multiplicação de vetores do NumPy é 
  usada para multiplicar o vetor de dados pelas colunas da matriz, construindo o vetor de retorno da DFT.
  \item \textbf{Matrix (NumPy)}: método um pouco mais inteligente de multiplicação pela matriz de DFT. A matriz DFT inteira é
  guardada como uma matriz do NumPy, e a aplicação da DFT consiste numa única operação de multiplicação matriz-vetor do NumPy.
\end{itemize}

Em todos meus testes, a \textit{FFT} foi sempre a mais rápida (quase instantânea), como esperado. No meu computador (CPU Core i7)
com uma versão recente do NumPy, o método \textit{Matrix(NumPy)} é bem eficiente, se aproximando do tempo do FFT, enquanto o método
\textit{Matrix} normal demora consideravelmente mais que os outros 2 métodos.

No entanto em outra máquina que testei (CPU Core i5) com uma versão mais antiga do NumPy, o método \textit{Matrix} foi mais rápido
que o método \textit{Matrix(NumPy)}, e nenhum dos dois chegou perto do \textit{FFT}.

Antes de testar os métodos de multiplicação pela matriz com os sons, realizei vários testes comparando os resultados da FFT com
os outros dois métodos, para garantir que estavam de acordo. Do jeito que estão implementados, pelo que vi os resultados entre os
dois métodos de multiplicação nunca diferem, e dependendo dos dados, o pior que notei foi diferirem em relação à FFT somente a
partir da sexta casa decimal.

Obs.: na contagem do tempo foi levado em consideração somente o tempo que levou para aplicar o método de DFT no conjunto de dados
passado, como a execução da chamada do FFT ou a multiplicação com a matriz DFT. A criação das matriz DFT não foi considerada.

\subsection{Vantagens e Desvantagens dos Métodos}

O FFT é sem duvida o mais eficiente. Tanto em velocidade como em gasto de memória. Porém é um algoritmo bem mais complexo e
consequentemente mais complicado para implementar.

Já o método da matriz é mais ingênuo. Dependendo de como você implementou poderá ter problemas em velocidade, gasto de memória
ou ambos. Em compensação é um algoritmo muito mais simples e fácil de entender, e há várias coisas que você pode fazer com o
algoritmo para mudar sua eficiência, como mostrei aqui com os dois métodos de multiplicação de matriz, mas é altamente improvável
que ele consiga ser tão eficiente quanto o FFT.

\section{Gráficos da Onda Domínio-Tempo e Domínio-Frequência}

O próprio EP cria os gráficos no término da transcrição musical e mostra eles para o usuário (vide README).

\end{document}
