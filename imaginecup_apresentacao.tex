\documentclass[brazil]{beamer}
\usepackage{beamerthemesplit}
\usepackage[brazilian]{babel}
\usepackage[utf8]{inputenc}
\usepackage{color}
\usepackage{xcolor}
\usepackage{fancybox}
\usepackage{ulem}
\usepackage{upquote}
\usetheme{JuanLesPins}

\title{USPGameDev - Sobre a ImagineCup}
\author{Fernando Omar Aluani}

%%%%%%%%%%%%%%%%%%%%%%%%%
% -o que é (resumo das competições e calendário) K
% -como funciona (fases local, online e tal) K
% -competições (falando um pouco das regras de cada uma) K
% -prêmios K
% -formação de equipes 
% -premios anteriores para equipes brasileiras


\begin{document}

\frame{\titlepage}

\section{Introdução}

\frame{
  \begin{center}
    \LARGE Introdução
  \end{center}
}

\frame{
  \underline{\Large O que é:}
  
  \pause
  \vspace{10pt}
  \hspace{10pt}
  É uma competição anual feita pela Microsoft, onde equipes de jovens do mundo afora criam
  software para um dentre os vários desafios que existem na competição.
}

\frame{
  \underline{\Large Competições esse ano:}
  
  \pause
  \vspace{10pt}
  \hspace{10pt}

  \begin{itemize}
    \item \textbf{Games}: fazer um jogo divertido, pode ser para várias plataformas (Windows, XBox, WinPhone, ...)
    \item \textbf{Innovation}: fazer um programa inovador.
    \item \textbf{World Citizenship}: Faça um programa que mude a vida de alguém. Escolha uma causa (meio ambiente,
    educação, saúde, etc), e use a tecnologia como um agente de mudança.
  \end{itemize}

}
\frame{
  \underline{\Large Desafios esse ano:}
  
  \pause
  \vspace{10pt}
  \hspace{10pt}

  \begin{itemize}
    \item Windows 8 App
    \item Windows Azure
    \item Windows Phone
  \end{itemize}
}

\frame{
  \underline{\Large Calendário:}
  
  \pause
  \vspace{10pt}
  \hspace{10pt}

  \textbf{30/Agosto/2012}: Início da Imagine Cup 2013
  \vspace{5pt}
  
  \textbf{15/Março/2013}: Limite para registrar as equipes.
  \vspace{5pt}

  \textbf{15/Abril/2013}: Finais locais terminam nos países;
    Data limite para submissão para as finais online.
  \vspace{5pt}


  \textbf{15/Maio/2013}: finalistas anunciados no site ImagineCup.com
  \vspace{5pt}

  \textbf{8-12/Julho/2013}: Finais mundiais em São Petersburgo, Russia.
    Worldwide Finals in St. Petersburg, Russia
}

\section{Como Funciona}
\frame{
  \begin{center}
    \LARGE Como Funciona?
  \end{center}
}

\frame{
  \underline{\Large No caso da competição de Games:}
  \pause
  \vspace{5pt}
  
  \textbf{\Large Round 1 - Finais Locais/Online: }
  \pause
  \vspace{5pt}
  
  Cada país que tiver uma competição local irá escolher \textbf{uma} equipe para
  representar o país, e tal equipe irá para a final mundial, na competição que ela
  escolheu.
  \vspace{5pt}
  
  Os vencedores das finais locais que não foram escolhidos para ser o time nacional
  são automaticamente inscritos na final online, ganhando uma segunda chance de chegar
  na final mundial.
  \vspace{5pt}
  
  As finais online escolhem 2 ou mais equipes de cada competição para ir para a final
  mundial.
  \vspace{5pt}
}

\frame{
  \textbf{\Large Round 2 - Finais Mundiais: }
  \pause
  \vspace{5pt}
  
  As equipes classificadas para a Final Mundial já ganham a viagem para a Rússia,
  hotel e algumas refeições durante o evento.
  \vspace{5pt}
  
  Lá será possível fazer modificações de última hora no jogo, e então os juízes irão
  escolher os 3 melhores jogos, baseados (em ordem de importância): diversão, execução,
  inovação e viabilidade de comercialização.
}

\frame{
  \textbf{\Large Prêmios: }
  \pause
  \vspace{5pt}
  
  \hspace{30pt}
  \begin{itemize}
    \item \textbf{Primeiro Colocado}: \$50 000 USD 
    \item \textbf{Segundo Colocado}: \$10 000 USD
    \item \textbf{Terceiro Colocado}: \$5 000 USD
  \end{itemize}
  \vspace{5pt}
  
  Bônus de \$1000 USD se o projeto incluir um \textbf{Windows 8 Store App}.
  \vspace{10pt}
  
  Todos esses prêmios são dados para a equipe, para serem divididos igualmente
  entre os integrantes. O mentor da equipe não recebe nenhum prêmio, exceto a viagem
  para a Final Mundial junto com a equipe caso tenham sido classificados.
}

\frame{
  \textbf{\Large Extra Gold no Round 1? }
  \pause
  \vspace{5pt}
  
  As finais locais no Round 1, em cada país, podem dar algum tipo de prêmio para 
  seus vencedores também.
  \vspace{5pt}
  
  Mas isso depende da subsidiária local da Microsoft cuidando da competição local,
  e enquanto eu tenho quase certeza que tem uma competição local aqui no Brasil,
  eu não consegui achar nada sobre ela...
}

\section{Equipes}

\frame{
  \begin{center}
    \LARGE Equipes
  \end{center}
}

\frame{
  A equipe pode ter de um a quatro integrantes, e opcionalmente um mentor. Para os
  integrantes é necessário:
  \begin{itemize}
    \item +16 anos (duh);
    \item Cursando colegial ou faculdade (duh again);
    \item Não estar trabalhando ou ter algum parente imediato que trabalhe na
          Microsoft ou uma subsidiária;
    \item Não estar envolvido ou ter algum parente imediato que esteja envolvido na
          administração da competição.
  \end{itemize}
}

\frame{  
  O mentor é alguma pessoa (tipo um professor) que guia os integrantes, dá um apoio moral.
  Mas não recebe os prêmios de \$, caso a equipe ganhe, e ele não deve ajudar diretamente
  no projeto (escrevendo código, por exemplo).
  \vspace{5pt}
  
  Cada integrante pode participar de mais de uma equipe, mas de somente uma por competição/desafio.
  Logo cada equipe pode participar de cada competição/desafio, mas o projeto para cada um deverá ser
  diferente.
}

\section{Vencedores Anteriores}

\frame{
  \begin{center}
    \LARGE Vencedores Anteriores
  \end{center}
}

\frame{  
  Brasil já teve várias equipes vencedoras nas Finais Mundiais (seriously), em um dos três primeiros
  lugares, entre várias das competições e desafios que tiveram nas Imagine Cup passadas.
  \vspace{5pt}
  
  Mais notavelmente, ganhamos \textbf{primeiro lugar na competição de Games (Game Design) em 2008, 2009 e 2011.}
  Ano passado, ganhamos segundo lugar em Game Design, e primeiro lugar em 3 dos quatro desafios (Kinect,
  Windows Azure e Windows Metro Style).
  
  Vi em algum lugar durante minha pesquisa que uma dessas equipes vencedoras do ano passado era da Poli...
}

\end{document}
