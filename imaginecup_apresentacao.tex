\documentclass[brazil]{beamer}
\usepackage{beamerthemesplit}
\usepackage[brazilian]{babel}
\usepackage[utf8]{inputenc}
\usepackage{color}
\usepackage{xcolor}
\usepackage{fancybox}
\usepackage{ulem}
\usepackage{upquote}
\usetheme{JuanLesPins}

\title{USPGameDev - Sobre a ImagineCup}
\author{Fernando Omar Aluani}

%%%%%%%%%%%%%%%%%%%%%%%%%
% -o que é (resumo das competições e calendário)
% -como funciona (fases local, online e tal)
% -competições (falando um pouco das regras de cada uma)
% -prêmios
% -formação de equipes
% -premios anteriores para equipes brasileiras


\begin{document}

\frame{\titlepage}

\section{Introdução}

\frame{
  \begin{center}
    \LARGE Introdução
  \end{center}
}

\frame{
  \underline{\Large O que é:}
  
  \pause
  \vspace{10pt}
  \hspace{10pt}
  É uma competição anual feita pela Microsoft, onde equipes de jovens do mundo afora criam
  software para um dentre os vários desafios que existem na competição.
}

\frame{
  \underline{\Large Competições esse ano:}
  
  \pause
  \vspace{10pt}
  \hspace{10pt}

  \begin{itemize}
    \item Games: fazer um jogo divertido, pode ser para várias plataformas (Windows, XBox, WinPhone, ...)
    \item Innovation: fazer um programa inovador.
    \item World Citizenship: Faça um programa que mude a vida de alguém. Escolha uma causa (meio ambiente,
    educação, saúde, etc), e use a tecnologia como um agente de mudança.
  \end{itemize}

}
\frame{
  \underline{\Large Desafios esse ano:}
  
  \pause
  \vspace{10pt}
  \hspace{10pt}

  \begin{itemize}
    \item Windows 8 App
    \item Windows Azure
    \item Windows Phone
  \end{itemize}
}

\frame{
  \underline{\Large Calendário:}
  
  \pause
  \vspace{10pt}
  \hspace{10pt}

  \textbf{30/Agosto/2012}: Início da Imagine Cup 2013
  \vspace{5pt}
  
  \textbf{15/Março/2013}: Limite para registrar as equipes.
  \vspace{5pt}

  \textbf{15/Abril/2013}: Finais locais terminam nos países;
    Data limite para submissão para as finais online.
  \vspace{5pt}


  \textbf{15/Maio/2013}: finalistas anunciados no site ImagineCup.com
  \vspace{5pt}

  \textbf{8-12/Julho/2013}: Finais mundiais em São Petersburgo, Russia.
    Worldwide Finals in St. Petersburg, Russia
}

\section{Como Funciona}
\frame{
  \begin{center}
    \LARGE Como Funciona?
  \end{center}
}

\frame{
  \underline{\Large No caso da competição de Games:}
  \pause
  \vspace{5pt}
  
  \textbf{\Large Round 1 - Finais Locais/Online: }
  \pause
  \vspace{5pt}
  
  Cada país que tiver uma competição local irá escolher \textbf{uma} equipe para
  representar o país, e tal equipe irá para a final mundial, na competição que ela
  escolheu.
  \vspace{5pt}
  
  Os vencedores das finais locais que não foram escolhidos para ser o time nacional
  são automaticamente inscritos na final online, ganhando uma segunda chance de chegar
  na final mundial.
  \vspace{5pt}
  
  As finais online escolhem 2 ou mais equipes de cada competição para ir para a final
  mundial.
  \vspace{5pt}
}

\frame{
  \textbf{\Large Round 2 - Finais Mundiais: }
  \pause
  \vspace{5pt}
  
  Cada país que tiver uma competição local irá escolher \textbf{uma} equipe para
  representar o país, e tal equipe irá para a final mundial, na competição que ela
  escolheu.
  \vspace{5pt}
  
  Os vencedores das finais locais que não foram escolhidos para ser o time nacional
  são automaticamente inscritos na final online, ganhando uma segunda chance de chegar
  na final mundial.
  \vspace{5pt}
  
  As finais online escolhem 2 ou mais equipes de cada competição para ir para a final
  mundial.
  \vspace{5pt}
}


\begin{frame}[fragile]
  \underline{\Large \textbf{Reduce}:}
  
  \vspace{4pt}
  \begin{itemize}
    \item[Entrada] Pares produzidos pelo \textbf{Map}.
    \item[Saida] Coleção de pares \verb$(Estatística, Valor)$.
  \end{itemize}
  
  \pause
  \hspace{10pt}
  Onde:
  \begin{itemize}
    \item \verb$Estatística$ é o nome de alguma estatística calculada a partir dos atributos do commit, 
          como por exemplo \verb$NomeDoAtributo_media$;
    \item \verb$Valor$ é o valor da estatística;
  \end{itemize}
\end{frame}


\end{document}
