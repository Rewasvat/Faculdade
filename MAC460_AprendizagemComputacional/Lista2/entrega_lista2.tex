\documentclass[a4paper,11pt]{article}
\usepackage[T1]{fontenc}
\usepackage[utf8]{inputenc}
\usepackage{lmodern}

\title{MAC460 - Entrega Lista 2}
\author{Fernando Omar Aluani (NUSP: 6797226)}

\begin{document}

\maketitle

\section{Exercício 5}

Para simular o algoritmo, escrevi um pequeno script Python (usando a biblioteca \textit{NumPy})
que ao ser rodado executa o algoritmo para o conjunto de amostras dadas e com valores para os
paramêtros $\eta$ e $\theta$ passados pela linha de comando. Incluí esse script, \textit{"perceptron.py"}, 
junto com este PDF com as respostas.

O $\eta$ serve para escalar o vetor $\Delta$, construido a partir da soma dos vetores de dados que foram
classificados erroneamente. Por sua vez, o $\Delta$ é somado ao $a$ para alterá-lo para a próxima iteração.

O $\theta$ é um valor usado como delimitador. Se $\left|\Delta\right| < \theta$, o algoritmo termina.
Se for um valor muito alto, é provável que o algoritmo termine sem achar a solução correta, Porém um valor
muito baixo (perto de 0 (positivo) ou qualquer negativo) pode fazer com que o algoritmo entre em loop infinito.

No caso específico desta entrada, para qualquer valor positivo de $\eta$ e $\theta$ só há duas possibilidades
de sequência de passos do algoritmo. Se $2\eta \geq \theta$, então o algoritmo irá terminar em 5 iterações,
sendo que na última $\left|\Delta\right| = 0$, e a solução estará correta. Caso contrário o algoritmo irá
terminar após a primeira iteração, com uma solução errônea.


\section{Exercício 7}

Usando Python e a biblioteca \textit{NumPy}, calculei os autovetores $v_{i}$ das matrizes, devidamente
ordenados em ordem decrescente pelos seus respectivos autovalores $\lambda_{i}$:

Para a matriz de covariância $\Sigma$:
$$v_{1} = (-0.04030552, -0.9991874) $$
$$\lambda_{1} = 100.16135318 $$
$$v_{2} = (-0.9991874 ,  0.04030552) $$
$$\lambda_{2} = 0.83864682 $$

Para a matriz de correlação $\rho $:
$$v_{1} = (0.70710678,  0.70710678) $$
$$\lambda_{1} = 1.4 $$
$$v_{2} = (-0.70710678,  0.70710678) $$
$$\lambda_{2} = 0.6 $$

Os componentes principais das matrizes são dados pelos autovetores das mesmas, e assim podemos
usá-los para comparar os componentes. 

Como é possível notar, existe uma grande diferença entre eles. Isso era esperado acontecer,
pois os componentes principais são baseados na variância das variáveis aleatórias, e no caso
da matriz de correlação, as variáveis são normalizadas para possuírem variância 1.

\section{Exercício 10}

Um ponto importante do artigo são seus comentários sobre como realizar estudos estatísticamente significativos,
notando os vários problemas encontrados em artigos que apresentam resultados de testes.

Mas achei que não ficou muito claro exatamente as explicações de (e argumentos usando) conceitos estatísticos. 


\section{Exercício 11}

Não tentei resolver os demais exercícios.

\end{document}
