\documentclass[a4paper,11pt]{article}
\usepackage[T1]{fontenc}
\usepackage[utf8]{inputenc}
\usepackage{lmodern}

\title{MAC460 - Entrega Lista 2}
\author{Fernando Omar Aluani (NUSP: 6797226)}

\begin{document}

\maketitle

\section{Exercício 5}

Para simular o algoritmo, escrevi um pequeno script Python (usando a biblioteca \textit{NumPy})
que ao ser rodado executa o algoritmo para o conjunto de amostras dadas e com valores para os
paramêtros $\eta$ e $\theta$ passados pela linha de comando. Incluí esse script, \textit{"perceptron.py"}, 
junto com este PDF com as respostas.




\section{Exercício 7}

Usando uma biblioteca matemática, calculei os autovetores $v_{i}$ das matrizes, devidamente
ordenados em ordem decrescente pelos seus respectivos autovalores $a_{i}$:

Para a matriz de covariância $\Sigma$:
$$v_{1} = (-0.04030552, -0.9991874) $$
$$a_{1} = 100.16135318 $$
$$v_{2} = (-0.9991874 ,  0.04030552) $$
$$a_{2} = 0.83864682 $$

Para a matriz de correlação $\rho $:
$$v_{1} = (0.70710678,  0.70710678) $$
$$a_{1} = 1.4 $$
$$v_{2} = (-0.70710678,  0.70710678) $$
$$a_{2} = 0.6 $$

Os componentes principais das matrizes são dados pelos autovetores das mesmas, e assim podemos
usá-los para comparar os componentes. 

Como é possível notar, existe uma grande diferença entre eles. Isso era esperado acontecer,
pois os componentes principais são baseados na variância das variáveis aleatórias, e no caso
da matriz de correlação, as variáveis são normalizadas para possuírem variância 1.

\section{Exercício 10}

Um ponto importante do artigo são seus comentários sobre como realizar estudos estatísticamente significativos,
notando os vários problemas encontrados em artigos que apresentam resultados de testes.

Mas achei que não ficou muito claro exatamente as explicações de (e argumentos usando) conceitos estatísticos. 


\section{Exercício 11}

Não tentei resolver os demais exercícios.

\end{document}